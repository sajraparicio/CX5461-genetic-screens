\documentclass[main]{subfiles}

ABSTRACT. 


DNA G-quadruplex (G4) formation at guanine-rich sequences can interfere with the normal progression of many cellular processes, including DNA replication, repair, and recombination. A number of factors involved in DNA repair are also required for the resolution of these structures. Deficiency of any of these factors results in genome instability at these sites. Many G4 ligands have been reported that promote genome instability at G4 sites in the absence of any of these factors. One such ligand, CX-5461, has been shown to increase cytotoxicity in BRCA1/2 deficient cancer cells and is currently being tested in phase I/II clinical trial for breast malignancies. To gain a comprehensive understanding of how G4 structures are resolved and to broaden the spectrum of tumors that could be targeted with G4 ligands, we conducted a pooled subgenomic CRISPR-Cas9 genetic screen in human colorectal cancer cell line HCT116. We used two G4 ligands, CX-5461 and pyridostatin (PDS), to examine gene-drug interactions with 480 genes including genome maintenance genes. Overall, there were many overlaps between the top hits from CX-5461 and PDS screens.  The list of top hits included genes previously known to resolve G4 structures. We also identified and validated genes and pathways that are not known to be involved in G4 structure resolution, including DNA damage response (DDR)-associated ubiquitin signaling factors, UBE2N and RNF168. We further characterized the mechanism of regulatory ubiquitin signaling in G4 structure resolution using cell-based assays. In conclusion, we have uncovered a novel mechanism of G4 structure resolution that involves chromatin ubiquitin modification.  These results also suggest that patients with UBE2N- and RNF168-mutated tumors could potentially benefit from CX-5461 therapy. In addition, combination therapy using inhibitors of DDR-associated ubiquitination could be used to enhance the effectiveness of CX-5461 in many cancers.
